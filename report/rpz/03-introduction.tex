\chapter*{ВВЕДЕНИЕ}
\addcontentsline{toc}{chapter}{ВВЕДЕНИЕ}

	Преломление света — это изменение направления распространения светового пучка при его переходе через границу двух сред с разными показателями преломления. Это явление лежит в основе работы множества оптических приборов, таких как линзы, которые находят широкое применение в фотографии, микроскопии, телескопах и многих других сферах. Классическим примером демонстрации преломления является прохождение света через выпуклую или вогнутую линзу.

	Современные методы моделирования и визуализации позволяют исследовать и воспроизводить это явление с высокой степенью точности. Программное обеспечение для визуализации преломления света через линзы позволяет исследовать и анализировать оптические системы, а так же создавать реалистичные эффекты в кинематографии.

	Целью данной курсовой является разработка программы для визуализации процесса преломления света через аналитически заданные линзы. Программа должна предоставить пользователю возможность выбора источников освещения и объектов, а также параметров сцены для анализа оптических эффектов.

	Для достижения поставленной цели необходимо выполнить следующие задачи:

\begin{itemize} 
	\item[--] изучение явления преломления света и законов оптики с физической точки зрения; 
	\item[--] анализ существующих алгоритмов визуализации оптических эффектов; 
	\item[--] выбор алгоритма для решения поставленной задачи; 
	\item[--] проектирование архитектуры программы и графического интерфейса; 
	\item[--] реализация структур данных и алгоритмов;
	\item[--] описание структуры программного обеспечения; 
	\item[--] реализация программы; 
	\item[--] исследование производительности программы. 
\end{itemize}