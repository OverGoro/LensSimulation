\chapter*{ЗАКЛЮЧЕНИЕ}
\addcontentsline{toc}{chapter}{ЗАКЛЮЧЕНИЕ}

В рамках курсовой работы было создано программное обеспечение, позволяющее визуализировать преломление света через оптические линзы. Разработанная система предоставляет пользователям возможность задавать материалы и другие параметры объектов, а также изменять положение точки наблюдения и характеристики источников света в процессе работы. В ходе выполнения проекта были успешно решены следующие задачи:
\begin{itemize}
    \item[--] рассмотрены физические основы явления;
    \item[--] проанализированы и выбраны модели представления трехмерных объектов;
    \item[--] выбран оптимальный алгоритм для решения поставленной задачи;
    \item[--] спроектирована архитектура программы и пользовательского интерфейса;
    \item[--] реализованы необходимые структуры данных и алгоритмы;
    \item[--] описана структура разработанного программного обеспечения;
    \item[--] продемонстрирована работоспособность программы;
    \item[--] проведено исследование производительности созданного ПО.
\end{itemize}
